% Options for packages loaded elsewhere
\PassOptionsToPackage{unicode}{hyperref}
\PassOptionsToPackage{hyphens}{url}
%
\documentclass[
]{article}
\usepackage{amsmath,amssymb}
\usepackage{lmodern}
\usepackage{iftex}
\ifPDFTeX
  \usepackage[T1]{fontenc}
  \usepackage[utf8]{inputenc}
  \usepackage{textcomp} % provide euro and other symbols
\else % if luatex or xetex
  \usepackage{unicode-math}
  \defaultfontfeatures{Scale=MatchLowercase}
  \defaultfontfeatures[\rmfamily]{Ligatures=TeX,Scale=1}
\fi
% Use upquote if available, for straight quotes in verbatim environments
\IfFileExists{upquote.sty}{\usepackage{upquote}}{}
\IfFileExists{microtype.sty}{% use microtype if available
  \usepackage[]{microtype}
  \UseMicrotypeSet[protrusion]{basicmath} % disable protrusion for tt fonts
}{}
\makeatletter
\@ifundefined{KOMAClassName}{% if non-KOMA class
  \IfFileExists{parskip.sty}{%
    \usepackage{parskip}
  }{% else
    \setlength{\parindent}{0pt}
    \setlength{\parskip}{6pt plus 2pt minus 1pt}}
}{% if KOMA class
  \KOMAoptions{parskip=half}}
\makeatother
\usepackage{xcolor}
\usepackage[margin=1in]{geometry}
\usepackage{color}
\usepackage{fancyvrb}
\newcommand{\VerbBar}{|}
\newcommand{\VERB}{\Verb[commandchars=\\\{\}]}
\DefineVerbatimEnvironment{Highlighting}{Verbatim}{commandchars=\\\{\}}
% Add ',fontsize=\small' for more characters per line
\usepackage{framed}
\definecolor{shadecolor}{RGB}{248,248,248}
\newenvironment{Shaded}{\begin{snugshade}}{\end{snugshade}}
\newcommand{\AlertTok}[1]{\textcolor[rgb]{0.94,0.16,0.16}{#1}}
\newcommand{\AnnotationTok}[1]{\textcolor[rgb]{0.56,0.35,0.01}{\textbf{\textit{#1}}}}
\newcommand{\AttributeTok}[1]{\textcolor[rgb]{0.77,0.63,0.00}{#1}}
\newcommand{\BaseNTok}[1]{\textcolor[rgb]{0.00,0.00,0.81}{#1}}
\newcommand{\BuiltInTok}[1]{#1}
\newcommand{\CharTok}[1]{\textcolor[rgb]{0.31,0.60,0.02}{#1}}
\newcommand{\CommentTok}[1]{\textcolor[rgb]{0.56,0.35,0.01}{\textit{#1}}}
\newcommand{\CommentVarTok}[1]{\textcolor[rgb]{0.56,0.35,0.01}{\textbf{\textit{#1}}}}
\newcommand{\ConstantTok}[1]{\textcolor[rgb]{0.00,0.00,0.00}{#1}}
\newcommand{\ControlFlowTok}[1]{\textcolor[rgb]{0.13,0.29,0.53}{\textbf{#1}}}
\newcommand{\DataTypeTok}[1]{\textcolor[rgb]{0.13,0.29,0.53}{#1}}
\newcommand{\DecValTok}[1]{\textcolor[rgb]{0.00,0.00,0.81}{#1}}
\newcommand{\DocumentationTok}[1]{\textcolor[rgb]{0.56,0.35,0.01}{\textbf{\textit{#1}}}}
\newcommand{\ErrorTok}[1]{\textcolor[rgb]{0.64,0.00,0.00}{\textbf{#1}}}
\newcommand{\ExtensionTok}[1]{#1}
\newcommand{\FloatTok}[1]{\textcolor[rgb]{0.00,0.00,0.81}{#1}}
\newcommand{\FunctionTok}[1]{\textcolor[rgb]{0.00,0.00,0.00}{#1}}
\newcommand{\ImportTok}[1]{#1}
\newcommand{\InformationTok}[1]{\textcolor[rgb]{0.56,0.35,0.01}{\textbf{\textit{#1}}}}
\newcommand{\KeywordTok}[1]{\textcolor[rgb]{0.13,0.29,0.53}{\textbf{#1}}}
\newcommand{\NormalTok}[1]{#1}
\newcommand{\OperatorTok}[1]{\textcolor[rgb]{0.81,0.36,0.00}{\textbf{#1}}}
\newcommand{\OtherTok}[1]{\textcolor[rgb]{0.56,0.35,0.01}{#1}}
\newcommand{\PreprocessorTok}[1]{\textcolor[rgb]{0.56,0.35,0.01}{\textit{#1}}}
\newcommand{\RegionMarkerTok}[1]{#1}
\newcommand{\SpecialCharTok}[1]{\textcolor[rgb]{0.00,0.00,0.00}{#1}}
\newcommand{\SpecialStringTok}[1]{\textcolor[rgb]{0.31,0.60,0.02}{#1}}
\newcommand{\StringTok}[1]{\textcolor[rgb]{0.31,0.60,0.02}{#1}}
\newcommand{\VariableTok}[1]{\textcolor[rgb]{0.00,0.00,0.00}{#1}}
\newcommand{\VerbatimStringTok}[1]{\textcolor[rgb]{0.31,0.60,0.02}{#1}}
\newcommand{\WarningTok}[1]{\textcolor[rgb]{0.56,0.35,0.01}{\textbf{\textit{#1}}}}
\usepackage{graphicx}
\makeatletter
\def\maxwidth{\ifdim\Gin@nat@width>\linewidth\linewidth\else\Gin@nat@width\fi}
\def\maxheight{\ifdim\Gin@nat@height>\textheight\textheight\else\Gin@nat@height\fi}
\makeatother
% Scale images if necessary, so that they will not overflow the page
% margins by default, and it is still possible to overwrite the defaults
% using explicit options in \includegraphics[width, height, ...]{}
\setkeys{Gin}{width=\maxwidth,height=\maxheight,keepaspectratio}
% Set default figure placement to htbp
\makeatletter
\def\fps@figure{htbp}
\makeatother
\setlength{\emergencystretch}{3em} % prevent overfull lines
\providecommand{\tightlist}{%
  \setlength{\itemsep}{0pt}\setlength{\parskip}{0pt}}
\setcounter{secnumdepth}{-\maxdimen} % remove section numbering
\ifLuaTeX
  \usepackage{selnolig}  % disable illegal ligatures
\fi
\IfFileExists{bookmark.sty}{\usepackage{bookmark}}{\usepackage{hyperref}}
\IfFileExists{xurl.sty}{\usepackage{xurl}}{} % add URL line breaks if available
\urlstyle{same} % disable monospaced font for URLs
\hypersetup{
  pdftitle={DATA605 - Homework 6},
  pdfauthor={Keith Colella},
  hidelinks,
  pdfcreator={LaTeX via pandoc}}

\title{DATA605 - Homework 6}
\author{Keith Colella}
\date{2023-10-08}

\begin{document}
\maketitle

\hypertarget{exercise-1}{%
\section{Exercise 1}\label{exercise-1}}

A bag contains 5 green and 7 red jellybeans. How many ways can 5
jellybeans be withdrawn from the bag so that the number of green ones
withdrawn will be less than 2?

\begin{Shaded}
\begin{Highlighting}[]
\NormalTok{n\_green }\OtherTok{\textless{}{-}} \DecValTok{5}
\NormalTok{n\_red }\OtherTok{\textless{}{-}} \DecValTok{7}

\NormalTok{no\_green }\OtherTok{\textless{}{-}} \FunctionTok{choose}\NormalTok{(n\_green, }\DecValTok{0}\NormalTok{) }\SpecialCharTok{*} \FunctionTok{choose}\NormalTok{(n\_red, }\DecValTok{5}\NormalTok{)}
\NormalTok{one\_green }\OtherTok{\textless{}{-}} \FunctionTok{choose}\NormalTok{(n\_green, }\DecValTok{1}\NormalTok{) }\SpecialCharTok{*} \FunctionTok{choose}\NormalTok{(n\_red, }\DecValTok{4}\NormalTok{)}

\NormalTok{no\_green }\SpecialCharTok{+}\NormalTok{ one\_green}
\end{Highlighting}
\end{Shaded}

\begin{verbatim}
## [1] 196
\end{verbatim}

\hypertarget{exercise-2}{%
\section{Exercise 2}\label{exercise-2}}

A certain congressional committee consists of 14 senators and 13
representatives. How many ways can a subcommittee of 5 be formed if at
least 4 of the members must be representatives?

\begin{Shaded}
\begin{Highlighting}[]
\NormalTok{n\_senators }\OtherTok{\textless{}{-}} \DecValTok{14}
\NormalTok{n\_reps }\OtherTok{\textless{}{-}} \DecValTok{13}

\NormalTok{all\_reps }\OtherTok{\textless{}{-}} \FunctionTok{choose}\NormalTok{(n\_reps, }\DecValTok{5}\NormalTok{)}
\NormalTok{one\_sen }\OtherTok{\textless{}{-}} \FunctionTok{choose}\NormalTok{(n\_reps, }\DecValTok{4}\NormalTok{) }\SpecialCharTok{*} \FunctionTok{choose}\NormalTok{(n\_senators, }\DecValTok{1}\NormalTok{)}

\NormalTok{all\_reps }\SpecialCharTok{+}\NormalTok{ one\_sen}
\end{Highlighting}
\end{Shaded}

\begin{verbatim}
## [1] 11297
\end{verbatim}

\hypertarget{exercise-3}{%
\section{Exercise 3}\label{exercise-3}}

If a coin is tossed 5 times, and then a standard six-sided die is rolled
2 times, and finally a group of three cards are drawn from a standard
deck of 52 cards without replacement, how many different outcomes are
possible?

\begin{Shaded}
\begin{Highlighting}[]
\NormalTok{coins }\OtherTok{\textless{}{-}} \DecValTok{2}\SpecialCharTok{\^{}}\DecValTok{5}
\NormalTok{dice }\OtherTok{\textless{}{-}} \DecValTok{6}\SpecialCharTok{\^{}}\DecValTok{2}
\NormalTok{cards }\OtherTok{\textless{}{-}} \FunctionTok{choose}\NormalTok{(}\DecValTok{52}\NormalTok{, }\DecValTok{3}\NormalTok{)}

\NormalTok{coins }\SpecialCharTok{*}\NormalTok{ dice }\SpecialCharTok{*}\NormalTok{ cards}
\end{Highlighting}
\end{Shaded}

\begin{verbatim}
## [1] 25459200
\end{verbatim}

\hypertarget{exercise-4}{%
\section{Exercise 4}\label{exercise-4}}

Three cards are drawn from a standard deck without replacement. What is
the probability that at least one of the cards drawn is a 3? Express
your answer as a fraction or a decimal number rounded to four decimal
places.

\hypertarget{solving-directly}{%
\subsection{Solving Directly}\label{solving-directly}}

\begin{Shaded}
\begin{Highlighting}[]
\NormalTok{p\_3\_x\_x }\OtherTok{\textless{}{-}} \DecValTok{4}\SpecialCharTok{/}\DecValTok{52} \SpecialCharTok{*}\NormalTok{ (}\DecValTok{51{-}3}\NormalTok{)}\SpecialCharTok{/}\DecValTok{51} \SpecialCharTok{*}\NormalTok{ (}\DecValTok{50{-}3}\NormalTok{)}\SpecialCharTok{/}\DecValTok{50}
\NormalTok{p\_x\_3\_x }\OtherTok{\textless{}{-}}\NormalTok{ (}\DecValTok{52{-}4}\NormalTok{)}\SpecialCharTok{/}\DecValTok{52} \SpecialCharTok{*} \DecValTok{4}\SpecialCharTok{/}\DecValTok{51} \SpecialCharTok{*}\NormalTok{ (}\DecValTok{50{-}3}\NormalTok{)}\SpecialCharTok{/}\DecValTok{50}
\NormalTok{p\_x\_x\_3 }\OtherTok{\textless{}{-}}\NormalTok{ (}\DecValTok{52{-}4}\NormalTok{)}\SpecialCharTok{/}\DecValTok{52} \SpecialCharTok{*}\NormalTok{ (}\DecValTok{51{-}4}\NormalTok{)}\SpecialCharTok{/}\DecValTok{51} \SpecialCharTok{*} \DecValTok{4}\SpecialCharTok{/}\DecValTok{50}
\NormalTok{p\_3\_3\_x }\OtherTok{\textless{}{-}} \DecValTok{4}\SpecialCharTok{/}\DecValTok{52} \SpecialCharTok{*} \DecValTok{3}\SpecialCharTok{/}\DecValTok{51} \SpecialCharTok{*}\NormalTok{ (}\DecValTok{50{-}2}\NormalTok{)}\SpecialCharTok{/}\DecValTok{50}
\NormalTok{p\_x\_3\_3 }\OtherTok{\textless{}{-}}\NormalTok{ (}\DecValTok{52{-}4}\NormalTok{)}\SpecialCharTok{/}\DecValTok{52} \SpecialCharTok{*} \DecValTok{4}\SpecialCharTok{/}\DecValTok{51} \SpecialCharTok{*} \DecValTok{3}\SpecialCharTok{/}\DecValTok{50}
\NormalTok{p\_3\_x\_3 }\OtherTok{\textless{}{-}} \DecValTok{4}\SpecialCharTok{/}\DecValTok{52} \SpecialCharTok{*}\NormalTok{ (}\DecValTok{51{-}3}\NormalTok{)}\SpecialCharTok{/}\DecValTok{51} \SpecialCharTok{*} \DecValTok{3}\SpecialCharTok{/}\DecValTok{50}
\NormalTok{p\_3\_3\_3 }\OtherTok{\textless{}{-}} \DecValTok{4}\SpecialCharTok{/}\DecValTok{52} \SpecialCharTok{*} \DecValTok{3}\SpecialCharTok{/}\DecValTok{51} \SpecialCharTok{*} \DecValTok{2}\SpecialCharTok{/}\DecValTok{50}

\FunctionTok{sum}\NormalTok{(}
\NormalTok{  p\_3\_x\_x,}
\NormalTok{  p\_x\_3\_x,}
\NormalTok{  p\_x\_x\_3,}
\NormalTok{  p\_3\_3\_x,}
\NormalTok{  p\_x\_3\_3,}
\NormalTok{  p\_3\_x\_3,}
\NormalTok{  p\_3\_3\_3}
\NormalTok{)}
\end{Highlighting}
\end{Shaded}

\begin{verbatim}
## [1] 0.2173756
\end{verbatim}

\hypertarget{solving-for-the-complement}{%
\subsection{Solving for the
Complement}\label{solving-for-the-complement}}

\begin{Shaded}
\begin{Highlighting}[]
\NormalTok{first }\OtherTok{\textless{}{-}}\NormalTok{ (}\DecValTok{52} \SpecialCharTok{{-}} \DecValTok{4}\NormalTok{) }\SpecialCharTok{/} \DecValTok{52}
\NormalTok{second }\OtherTok{\textless{}{-}}\NormalTok{ (}\DecValTok{51} \SpecialCharTok{{-}} \DecValTok{4}\NormalTok{) }\SpecialCharTok{/} \DecValTok{51}
\NormalTok{third }\OtherTok{\textless{}{-}}\NormalTok{ (}\DecValTok{50} \SpecialCharTok{{-}} \DecValTok{4}\NormalTok{) }\SpecialCharTok{/} \DecValTok{50}

\NormalTok{p\_no\_threes }\OtherTok{\textless{}{-}}\NormalTok{ first }\SpecialCharTok{*}\NormalTok{ second }\SpecialCharTok{*}\NormalTok{ third}

\DecValTok{1} \SpecialCharTok{{-}}\NormalTok{ p\_no\_threes}
\end{Highlighting}
\end{Shaded}

\begin{verbatim}
## [1] 0.2173756
\end{verbatim}

\hypertarget{exercise-5}{%
\section{Exercise 5}\label{exercise-5}}

Lorenzo is picking out some movies to rent, and he is primarily
interested in documentaries and mysteries. He has narrowed down his
selections to 17 documentaries and 14 mysteries.

\hypertarget{step-1}{%
\subsection{Step 1}\label{step-1}}

How many different combinations of 5 movies can he rent?

\begin{Shaded}
\begin{Highlighting}[]
\NormalTok{n\_docs }\OtherTok{\textless{}{-}} \DecValTok{17}
\NormalTok{n\_myst }\OtherTok{\textless{}{-}} \DecValTok{14}

\FunctionTok{choose}\NormalTok{(n\_docs }\SpecialCharTok{+}\NormalTok{ n\_myst, }\DecValTok{5}\NormalTok{)}
\end{Highlighting}
\end{Shaded}

\begin{verbatim}
## [1] 169911
\end{verbatim}

\hypertarget{step-2}{%
\subsection{Step 2}\label{step-2}}

How many different combinations of 5 movies can he rent if he wants at
least one mystery?

\begin{Shaded}
\begin{Highlighting}[]
\NormalTok{one\_myst }\OtherTok{\textless{}{-}} \FunctionTok{choose}\NormalTok{(n\_myst, }\DecValTok{1}\NormalTok{) }\SpecialCharTok{*} \FunctionTok{choose}\NormalTok{(n\_docs, }\DecValTok{4}\NormalTok{)}
\NormalTok{two\_myst }\OtherTok{\textless{}{-}} \FunctionTok{choose}\NormalTok{(n\_myst, }\DecValTok{2}\NormalTok{) }\SpecialCharTok{*} \FunctionTok{choose}\NormalTok{(n\_docs, }\DecValTok{3}\NormalTok{)}
\NormalTok{three\_myst }\OtherTok{\textless{}{-}} \FunctionTok{choose}\NormalTok{(n\_myst, }\DecValTok{3}\NormalTok{) }\SpecialCharTok{*} \FunctionTok{choose}\NormalTok{(n\_docs, }\DecValTok{2}\NormalTok{)}
\NormalTok{four\_myst }\OtherTok{\textless{}{-}} \FunctionTok{choose}\NormalTok{(n\_myst, }\DecValTok{4}\NormalTok{) }\SpecialCharTok{*} \FunctionTok{choose}\NormalTok{(n\_docs, }\DecValTok{1}\NormalTok{)}
\NormalTok{five\_myst }\OtherTok{\textless{}{-}} \FunctionTok{choose}\NormalTok{(n\_myst, }\DecValTok{5}\NormalTok{)}

\FunctionTok{sum}\NormalTok{(one\_myst, two\_myst, three\_myst, four\_myst, five\_myst)}
\end{Highlighting}
\end{Shaded}

\begin{verbatim}
## [1] 163723
\end{verbatim}

\hypertarget{exercise-6}{%
\section{Exercise 6}\label{exercise-6}}

In choosing what music to play at a charity fund raising event, Cory
needs to have an equal number of symphonies from Brahms, Haydn, and
Mendelssohn. If he is setting up a schedule of the 9 symphonies to be
played, and he has 4 Brahms, 104 Haydn, and 17 Mendelssohn symphonies
from which to choose, how many different schedules are possible? Express
your answer in scientific notation rounding to the hundredths place.

\begin{Shaded}
\begin{Highlighting}[]
\NormalTok{n\_brahms }\OtherTok{\textless{}{-}} \DecValTok{4}
\NormalTok{n\_hadyn }\OtherTok{\textless{}{-}} \DecValTok{104}
\NormalTok{n\_mendel }\OtherTok{\textless{}{-}} \DecValTok{17}

\NormalTok{song\_schedule }\OtherTok{\textless{}{-}} \FunctionTok{choose}\NormalTok{(n\_brahms, }\DecValTok{3}\NormalTok{) }\SpecialCharTok{*} 
  \FunctionTok{choose}\NormalTok{(n\_hadyn, }\DecValTok{3}\NormalTok{) }\SpecialCharTok{*} 
  \FunctionTok{choose}\NormalTok{(n\_mendel, }\DecValTok{3}\NormalTok{) }\SpecialCharTok{*} 
  \FunctionTok{factorial}\NormalTok{(}\DecValTok{9}\NormalTok{)}

\FunctionTok{formatC}\NormalTok{(song\_schedule, }\AttributeTok{format=}\StringTok{"e"}\NormalTok{, }\AttributeTok{digits=}\DecValTok{2}\NormalTok{)}
\end{Highlighting}
\end{Shaded}

\begin{verbatim}
## [1] "1.80e+14"
\end{verbatim}

\hypertarget{exercise-7}{%
\section{Exercise 7}\label{exercise-7}}

An English teacher needs to pick 13 books to put on his reading list for
the next school year, and he needs to plan the order in which they
should be read. He has narrowed down his choices to 6 novels, 6 plays, 7
poetry books, and 5 nonfiction books.

\begin{Shaded}
\begin{Highlighting}[]
\NormalTok{n\_novels }\OtherTok{\textless{}{-}} \DecValTok{6}
\NormalTok{n\_plays }\OtherTok{\textless{}{-}} \DecValTok{6}
\NormalTok{n\_poems }\OtherTok{\textless{}{-}} \DecValTok{7}
\NormalTok{n\_nonfict }\OtherTok{\textless{}{-}} \DecValTok{5}
\end{Highlighting}
\end{Shaded}

\hypertarget{step-1-1}{%
\subsection{Step 1}\label{step-1-1}}

If he wants to include no more than 4 nonfiction books, how many
different reading schedules are possible? Express your answer in
scientific notation rounding to the hundredths place.

\begin{Shaded}
\begin{Highlighting}[]
\NormalTok{n\_other }\OtherTok{\textless{}{-}}\NormalTok{ n\_novels }\SpecialCharTok{+}\NormalTok{ n\_plays }\SpecialCharTok{+}\NormalTok{ n\_poems}

\NormalTok{reading\_schedule1 }\OtherTok{\textless{}{-}} \FunctionTok{sum}\NormalTok{(}
  \FunctionTok{choose}\NormalTok{(n\_nonfict, }\DecValTok{4}\NormalTok{) }\SpecialCharTok{*} \FunctionTok{choose}\NormalTok{(n\_other, }\DecValTok{13} \SpecialCharTok{{-}} \DecValTok{4}\NormalTok{) }\SpecialCharTok{*} \FunctionTok{factorial}\NormalTok{(}\DecValTok{13}\NormalTok{),}
  \FunctionTok{choose}\NormalTok{(n\_nonfict, }\DecValTok{3}\NormalTok{) }\SpecialCharTok{*} \FunctionTok{choose}\NormalTok{(n\_other, }\DecValTok{13} \SpecialCharTok{{-}} \DecValTok{3}\NormalTok{) }\SpecialCharTok{*} \FunctionTok{factorial}\NormalTok{(}\DecValTok{13}\NormalTok{),}
  \FunctionTok{choose}\NormalTok{(n\_nonfict, }\DecValTok{2}\NormalTok{) }\SpecialCharTok{*} \FunctionTok{choose}\NormalTok{(n\_other, }\DecValTok{13} \SpecialCharTok{{-}} \DecValTok{2}\NormalTok{) }\SpecialCharTok{*} \FunctionTok{factorial}\NormalTok{(}\DecValTok{13}\NormalTok{),}
  \FunctionTok{choose}\NormalTok{(n\_nonfict, }\DecValTok{1}\NormalTok{) }\SpecialCharTok{*} \FunctionTok{choose}\NormalTok{(n\_other, }\DecValTok{13} \SpecialCharTok{{-}} \DecValTok{1}\NormalTok{) }\SpecialCharTok{*} \FunctionTok{factorial}\NormalTok{(}\DecValTok{13}\NormalTok{),}
  \FunctionTok{choose}\NormalTok{(n\_other, }\DecValTok{13}\NormalTok{) }\SpecialCharTok{*} \FunctionTok{factorial}\NormalTok{(}\DecValTok{13}\NormalTok{)}
\NormalTok{)}

\FunctionTok{formatC}\NormalTok{(reading\_schedule1, }\AttributeTok{format=}\StringTok{"e"}\NormalTok{, }\AttributeTok{digits=}\DecValTok{2}\NormalTok{)}
\end{Highlighting}
\end{Shaded}

\begin{verbatim}
## [1] "1.51e+16"
\end{verbatim}

\hypertarget{step-2-1}{%
\subsection{Step 2}\label{step-2-1}}

If he wants to include all 6 plays, how many different reading schedules
are possible? Express your answer in scientific notation rounding to the
hundredths place.

\begin{Shaded}
\begin{Highlighting}[]
\NormalTok{n\_not\_play }\OtherTok{\textless{}{-}}\NormalTok{ n\_novels }\SpecialCharTok{+}\NormalTok{ n\_poems }\SpecialCharTok{+}\NormalTok{ n\_nonfict}

\NormalTok{reading\_schedule2 }\OtherTok{\textless{}{-}} \FunctionTok{choose}\NormalTok{(n\_not\_play, }\DecValTok{13}\SpecialCharTok{{-}} \DecValTok{6}\NormalTok{) }\SpecialCharTok{*} \FunctionTok{factorial}\NormalTok{(}\DecValTok{13}\NormalTok{)}

\FunctionTok{formatC}\NormalTok{(reading\_schedule2, }\AttributeTok{format=}\StringTok{"e"}\NormalTok{, }\AttributeTok{digits=}\DecValTok{2}\NormalTok{)}
\end{Highlighting}
\end{Shaded}

\begin{verbatim}
## [1] "1.98e+14"
\end{verbatim}

\hypertarget{exercise-8}{%
\section{Exercise 8}\label{exercise-8}}

Zane is planting trees along his driveway, and he has 5 sycamores and 5
cypress trees to plant in one row. What is the probability that he
randomly plants the trees so that all 5 sycamores are next to each other
and all 5 cypress trees are next to each other? Express your answer as a
fraction or a decimal number rounded to four decimal places.

\begin{Shaded}
\begin{Highlighting}[]
\NormalTok{total\_plant\_orders }\OtherTok{\textless{}{-}} \FunctionTok{factorial}\NormalTok{(}\DecValTok{10}\NormalTok{)}

\NormalTok{grouped\_plant\_orders }\OtherTok{\textless{}{-}} \FunctionTok{factorial}\NormalTok{(}\DecValTok{2}\NormalTok{) }\SpecialCharTok{*} \FunctionTok{factorial}\NormalTok{(}\DecValTok{5}\NormalTok{) }\SpecialCharTok{*} \FunctionTok{factorial}\NormalTok{(}\DecValTok{5}\NormalTok{)}

\NormalTok{p\_grouped\_plants }\OtherTok{\textless{}{-}}\NormalTok{ grouped\_plant\_orders }\SpecialCharTok{/}\NormalTok{ total\_plant\_orders}

\FunctionTok{round}\NormalTok{(p\_grouped\_plants, }\DecValTok{4}\NormalTok{)}
\end{Highlighting}
\end{Shaded}

\begin{verbatim}
## [1] 0.0079
\end{verbatim}

\hypertarget{exercise-9}{%
\section{Exercise 9}\label{exercise-9}}

If you draw a queen or lower from a standard deck of cards, I will pay
you \$4. If not, you pay me \$16. (Aces are considered the highest card
in the deck.)

\hypertarget{step-1-2}{%
\subsection{Step 1}\label{step-1-2}}

Find the expected value of the proposition. Round your answer to two
decimal places. Losses must be expressed as negative values.

\begin{Shaded}
\begin{Highlighting}[]
\NormalTok{win }\OtherTok{\textless{}{-}}\NormalTok{ (}\DecValTok{52{-}8}\NormalTok{)}\SpecialCharTok{/}\DecValTok{52} \SpecialCharTok{*} \DecValTok{4}
\NormalTok{lose }\OtherTok{\textless{}{-}} \DecValTok{8}\SpecialCharTok{/}\DecValTok{52} \SpecialCharTok{*} \DecValTok{16}

\NormalTok{expected\_value }\OtherTok{\textless{}{-}}\NormalTok{ win }\SpecialCharTok{{-}}\NormalTok{ lose}

\FunctionTok{round}\NormalTok{(expected\_value, }\DecValTok{2}\NormalTok{)}
\end{Highlighting}
\end{Shaded}

\begin{verbatim}
## [1] 0.92
\end{verbatim}

\hypertarget{step-2-2}{%
\subsection{Step 2}\label{step-2-2}}

If you played this game 833 times how much would you expect to win or
lose? Round your answer to two decimal places. Losses must be expressed
as negative values.

\begin{Shaded}
\begin{Highlighting}[]
\NormalTok{total\_winnings }\OtherTok{\textless{}{-}}\NormalTok{ expected\_value }\SpecialCharTok{*} \DecValTok{833}

\FunctionTok{round}\NormalTok{(total\_winnings, }\DecValTok{2}\NormalTok{)}
\end{Highlighting}
\end{Shaded}

\begin{verbatim}
## [1] 768.92
\end{verbatim}

\end{document}
